% Options for packages loaded elsewhere
\PassOptionsToPackage{unicode}{hyperref}
\PassOptionsToPackage{hyphens}{url}
%
\documentclass[
]{article}
\usepackage{lmodern}
\usepackage{amssymb,amsmath}
\usepackage{ifxetex,ifluatex}
\ifnum 0\ifxetex 1\fi\ifluatex 1\fi=0 % if pdftex
  \usepackage[T1]{fontenc}
  \usepackage[utf8]{inputenc}
  \usepackage{textcomp} % provide euro and other symbols
\else % if luatex or xetex
  \usepackage{unicode-math}
  \defaultfontfeatures{Scale=MatchLowercase}
  \defaultfontfeatures[\rmfamily]{Ligatures=TeX,Scale=1}
\fi
% Use upquote if available, for straight quotes in verbatim environments
\IfFileExists{upquote.sty}{\usepackage{upquote}}{}
\IfFileExists{microtype.sty}{% use microtype if available
  \usepackage[]{microtype}
  \UseMicrotypeSet[protrusion]{basicmath} % disable protrusion for tt fonts
}{}
\makeatletter
\@ifundefined{KOMAClassName}{% if non-KOMA class
  \IfFileExists{parskip.sty}{%
    \usepackage{parskip}
  }{% else
    \setlength{\parindent}{0pt}
    \setlength{\parskip}{6pt plus 2pt minus 1pt}}
}{% if KOMA class
  \KOMAoptions{parskip=half}}
\makeatother
\usepackage{xcolor}
\IfFileExists{xurl.sty}{\usepackage{xurl}}{} % add URL line breaks if available
\IfFileExists{bookmark.sty}{\usepackage{bookmark}}{\usepackage{hyperref}}
\hypersetup{
  pdftitle={R Notebook},
  hidelinks,
  pdfcreator={LaTeX via pandoc}}
\urlstyle{same} % disable monospaced font for URLs
\usepackage[margin=1in]{geometry}
\usepackage{color}
\usepackage{fancyvrb}
\newcommand{\VerbBar}{|}
\newcommand{\VERB}{\Verb[commandchars=\\\{\}]}
\DefineVerbatimEnvironment{Highlighting}{Verbatim}{commandchars=\\\{\}}
% Add ',fontsize=\small' for more characters per line
\usepackage{framed}
\definecolor{shadecolor}{RGB}{248,248,248}
\newenvironment{Shaded}{\begin{snugshade}}{\end{snugshade}}
\newcommand{\AlertTok}[1]{\textcolor[rgb]{0.94,0.16,0.16}{#1}}
\newcommand{\AnnotationTok}[1]{\textcolor[rgb]{0.56,0.35,0.01}{\textbf{\textit{#1}}}}
\newcommand{\AttributeTok}[1]{\textcolor[rgb]{0.77,0.63,0.00}{#1}}
\newcommand{\BaseNTok}[1]{\textcolor[rgb]{0.00,0.00,0.81}{#1}}
\newcommand{\BuiltInTok}[1]{#1}
\newcommand{\CharTok}[1]{\textcolor[rgb]{0.31,0.60,0.02}{#1}}
\newcommand{\CommentTok}[1]{\textcolor[rgb]{0.56,0.35,0.01}{\textit{#1}}}
\newcommand{\CommentVarTok}[1]{\textcolor[rgb]{0.56,0.35,0.01}{\textbf{\textit{#1}}}}
\newcommand{\ConstantTok}[1]{\textcolor[rgb]{0.00,0.00,0.00}{#1}}
\newcommand{\ControlFlowTok}[1]{\textcolor[rgb]{0.13,0.29,0.53}{\textbf{#1}}}
\newcommand{\DataTypeTok}[1]{\textcolor[rgb]{0.13,0.29,0.53}{#1}}
\newcommand{\DecValTok}[1]{\textcolor[rgb]{0.00,0.00,0.81}{#1}}
\newcommand{\DocumentationTok}[1]{\textcolor[rgb]{0.56,0.35,0.01}{\textbf{\textit{#1}}}}
\newcommand{\ErrorTok}[1]{\textcolor[rgb]{0.64,0.00,0.00}{\textbf{#1}}}
\newcommand{\ExtensionTok}[1]{#1}
\newcommand{\FloatTok}[1]{\textcolor[rgb]{0.00,0.00,0.81}{#1}}
\newcommand{\FunctionTok}[1]{\textcolor[rgb]{0.00,0.00,0.00}{#1}}
\newcommand{\ImportTok}[1]{#1}
\newcommand{\InformationTok}[1]{\textcolor[rgb]{0.56,0.35,0.01}{\textbf{\textit{#1}}}}
\newcommand{\KeywordTok}[1]{\textcolor[rgb]{0.13,0.29,0.53}{\textbf{#1}}}
\newcommand{\NormalTok}[1]{#1}
\newcommand{\OperatorTok}[1]{\textcolor[rgb]{0.81,0.36,0.00}{\textbf{#1}}}
\newcommand{\OtherTok}[1]{\textcolor[rgb]{0.56,0.35,0.01}{#1}}
\newcommand{\PreprocessorTok}[1]{\textcolor[rgb]{0.56,0.35,0.01}{\textit{#1}}}
\newcommand{\RegionMarkerTok}[1]{#1}
\newcommand{\SpecialCharTok}[1]{\textcolor[rgb]{0.00,0.00,0.00}{#1}}
\newcommand{\SpecialStringTok}[1]{\textcolor[rgb]{0.31,0.60,0.02}{#1}}
\newcommand{\StringTok}[1]{\textcolor[rgb]{0.31,0.60,0.02}{#1}}
\newcommand{\VariableTok}[1]{\textcolor[rgb]{0.00,0.00,0.00}{#1}}
\newcommand{\VerbatimStringTok}[1]{\textcolor[rgb]{0.31,0.60,0.02}{#1}}
\newcommand{\WarningTok}[1]{\textcolor[rgb]{0.56,0.35,0.01}{\textbf{\textit{#1}}}}
\usepackage{graphicx,grffile}
\makeatletter
\def\maxwidth{\ifdim\Gin@nat@width>\linewidth\linewidth\else\Gin@nat@width\fi}
\def\maxheight{\ifdim\Gin@nat@height>\textheight\textheight\else\Gin@nat@height\fi}
\makeatother
% Scale images if necessary, so that they will not overflow the page
% margins by default, and it is still possible to overwrite the defaults
% using explicit options in \includegraphics[width, height, ...]{}
\setkeys{Gin}{width=\maxwidth,height=\maxheight,keepaspectratio}
% Set default figure placement to htbp
\makeatletter
\def\fps@figure{htbp}
\makeatother
\setlength{\emergencystretch}{3em} % prevent overfull lines
\providecommand{\tightlist}{%
  \setlength{\itemsep}{0pt}\setlength{\parskip}{0pt}}
\setcounter{secnumdepth}{-\maxdimen} % remove section numbering

\title{R Notebook}
\author{}
\date{\vspace{-2.5em}}

\begin{document}
\maketitle

This is an \href{http://rmarkdown.rstudio.com}{R Markdown} Notebook.
When you execute code within the notebook, the results appear beneath
the code.

Try executing this chunk by clicking the \emph{Run} button within the
chunk or by placing your cursor inside it and pressing
\emph{Ctrl+Shift+Enter}. Add a new chunk by clicking the \emph{Insert
Chunk} button on the toolbar or by pressing \emph{Ctrl+Alt+I}.

Problem 1 generates random values from the dice random variable

\begin{Shaded}
\begin{Highlighting}[]
\NormalTok{mydice <-}\StringTok{ }\ControlFlowTok{function}\NormalTok{(R)\{}
\NormalTok{out <-}\StringTok{ }\KeywordTok{rep}\NormalTok{(}\OtherTok{NA}\NormalTok{, R) }\CommentTok{# initialize output vector of length R}
\ControlFlowTok{for}\NormalTok{ (i }\ControlFlowTok{in} \DecValTok{1}\OperatorTok{:}\NormalTok{R)\{}
\NormalTok{va<-}\StringTok{ }\KeywordTok{runif}\NormalTok{(}\DecValTok{1}\NormalTok{,}\DecValTok{1}\NormalTok{,}\DecValTok{7}\NormalTok{)}
\ControlFlowTok{if}\NormalTok{ (va }\OperatorTok{==}\StringTok{ }\DecValTok{7}\NormalTok{)\{}
\ControlFlowTok{while}\NormalTok{(va}\OperatorTok{==}\DecValTok{7}\NormalTok{)\{}
\NormalTok{  va<-}\StringTok{ }\KeywordTok{runif}\NormalTok{(}\DecValTok{1}\NormalTok{,}\DecValTok{1}\NormalTok{,}\DecValTok{7}\NormalTok{)}
\NormalTok{\}}
\NormalTok{\}}
\NormalTok{var <-}\StringTok{ }\KeywordTok{floor}\NormalTok{(va)}
\NormalTok{out[i] <-}\StringTok{ }\NormalTok{var}
\NormalTok{\}}
\KeywordTok{return}\NormalTok{(out)}
\NormalTok{\}}
\KeywordTok{set.seed}\NormalTok{(}\DecValTok{1}\NormalTok{)}
\KeywordTok{system.time}\NormalTok{(mysample <-}\StringTok{ }\KeywordTok{mydice}\NormalTok{(}\DecValTok{100000}\NormalTok{))}
\end{Highlighting}
\end{Shaded}

\begin{verbatim}
##    user  system elapsed 
##    0.36    0.02    0.41
\end{verbatim}

\begin{Shaded}
\begin{Highlighting}[]
\NormalTok{(Freq <-}\StringTok{ }\KeywordTok{table}\NormalTok{(mysample)}\OperatorTok{/}\DecValTok{100000}\NormalTok{)}
\end{Highlighting}
\end{Shaded}

\begin{verbatim}
## mysample
##       1       2       3       4       5       6 
## 0.16913 0.16485 0.16610 0.16616 0.16735 0.16641
\end{verbatim}

\begin{Shaded}
\begin{Highlighting}[]
\NormalTok{(Freq <-}\StringTok{ }\KeywordTok{table}\NormalTok{(mysample))}
\end{Highlighting}
\end{Shaded}

\begin{verbatim}
## mysample
##     1     2     3     4     5     6 
## 16913 16485 16610 16616 16735 16641
\end{verbatim}

\begin{Shaded}
\begin{Highlighting}[]
\KeywordTok{barplot}\NormalTok{(Freq}\OperatorTok{/}\DecValTok{100000}\NormalTok{, }\DataTypeTok{ylim=} \KeywordTok{c}\NormalTok{(}\DecValTok{0}\NormalTok{,}\DecValTok{1}\NormalTok{),   }
\DataTypeTok{main =} \StringTok{"Observed relative frequences"}\NormalTok{)}
\end{Highlighting}
\end{Shaded}

\includegraphics{compstats1_files/figure-latex/unnamed-chunk-1-1.pdf}

Problem 2 generates random values from a Exp(λ)-distribution using the
probability integral transform method

Lemma 1.8 (Probability integral transform). If X is a continuous random
variable with distribution function F(·) and inverse F\^{}−1(·), then U
:= F(X) ∼ U{[}0, 1{]}. Therefore, if X := F\^{}−1(U), then X ∼ F. The
transformation U := F(X) is called the probability integral transform
(PIT) of X

Since F(x) = 1 − exp(−λx), we have F\^{}−1(u) = −(log(1−u))/λ. Note that
1 − U has the same distribution as U, if U ∼ U{[}0, 1{]}, therefore
−(log(1−u))/λ leads to a value from Exp(λ)

\begin{Shaded}
\begin{Highlighting}[]
\CommentTok{#our PIT generated exponential function}
\NormalTok{myexp <-}\StringTok{ }\ControlFlowTok{function}\NormalTok{(R,lambda)\{}
\NormalTok{out <-}\StringTok{ }\KeywordTok{rep}\NormalTok{(}\OtherTok{NA}\NormalTok{, R) }\CommentTok{# initialize output vector of length R}
\ControlFlowTok{for}\NormalTok{ (i }\ControlFlowTok{in} \DecValTok{1}\OperatorTok{:}\NormalTok{R)\{}
\NormalTok{va<-}\StringTok{ }\KeywordTok{runif}\NormalTok{(}\DecValTok{1}\NormalTok{)}
\NormalTok{var<-}\StringTok{ }\OperatorTok{-}\NormalTok{(}\KeywordTok{log}\NormalTok{(1−va))}\OperatorTok{/}\NormalTok{lambda}
\NormalTok{out[i] <-}\StringTok{ }\NormalTok{var}
\NormalTok{\}}
\KeywordTok{return}\NormalTok{(out)}
\NormalTok{\}}

\CommentTok{#generates true exponential function}
\NormalTok{trueexp <-}\StringTok{ }\ControlFlowTok{function}\NormalTok{(R,lambda)\{}
\NormalTok{out <-}\StringTok{ }\KeywordTok{rep}\NormalTok{(}\OtherTok{NA}\NormalTok{, R) }\CommentTok{# initialize output vector of length R}
\ControlFlowTok{for}\NormalTok{ (i }\ControlFlowTok{in} \DecValTok{1}\OperatorTok{:}\NormalTok{R)\{}
\NormalTok{var<-}\StringTok{ }\KeywordTok{dexp}\NormalTok{(lambda)}
\NormalTok{out[i] <-}\StringTok{ }\NormalTok{var}
\NormalTok{\}}
\KeywordTok{return}\NormalTok{(out)}
\NormalTok{\}}

\KeywordTok{set.seed}\NormalTok{(}\DecValTok{1}\NormalTok{)}
\NormalTok{lambd<-}\StringTok{ }\DecValTok{1}

\KeywordTok{system.time}\NormalTok{(mysample <-}\StringTok{ }\KeywordTok{myexp}\NormalTok{(}\DecValTok{100000}\NormalTok{,lambd))}
\end{Highlighting}
\end{Shaded}

\begin{verbatim}
##    user  system elapsed 
##    0.25    0.00    0.26
\end{verbatim}

\begin{Shaded}
\begin{Highlighting}[]
\KeywordTok{system.time}\NormalTok{(texp <-}\StringTok{ }\KeywordTok{trueexp}\NormalTok{(}\DecValTok{100000}\NormalTok{,lambd))}
\end{Highlighting}
\end{Shaded}

\begin{verbatim}
##    user  system elapsed 
##    0.16    0.00    0.24
\end{verbatim}

\begin{Shaded}
\begin{Highlighting}[]
\CommentTok{# Kernel Density Plot}
\NormalTok{d <-}\StringTok{ }\KeywordTok{density}\NormalTok{(mysample) }\CommentTok{# returns the density data from our PIT method  }
\NormalTok{dt <-}\StringTok{ }\KeywordTok{density}\NormalTok{(texp)}\CommentTok{#return true density function}
\KeywordTok{plot}\NormalTok{(d,}\DataTypeTok{ylim=}\KeywordTok{c}\NormalTok{(}\DecValTok{0}\NormalTok{,}\DecValTok{12}\NormalTok{),}\DataTypeTok{col=}\StringTok{"red"}\NormalTok{,}\DataTypeTok{main=}\StringTok{"Kernel Density of Exp(1)-distribution"}\NormalTok{)}
\KeywordTok{lines}\NormalTok{(dt,}\DataTypeTok{col=}\StringTok{"blue"}\NormalTok{)}
\KeywordTok{legend}\NormalTok{(}\StringTok{"topright"}\NormalTok{,}\KeywordTok{c}\NormalTok{(}\StringTok{"Sample"}\NormalTok{,}\StringTok{"True"}\NormalTok{),}\DataTypeTok{fill=}\KeywordTok{c}\NormalTok{(}\StringTok{"red"}\NormalTok{,}\StringTok{"blue"}\NormalTok{))}
\end{Highlighting}
\end{Shaded}

\includegraphics{compstats1_files/figure-latex/unnamed-chunk-2-1.pdf}

Problem 3 Let N(t) be a Poisson process with parameter λ. Using the
properties (i)-(iii) given in the lecture (Definition 1.6 in the script)
prove that N(t + s) − N(t) ∼ Poi(λs). Hint: The characteristic function
for X ∼ Poi(λ) is φX(u) = exp(λ(exp(iu) − 1)).

N(t + s) ∼ Poi(λ(t+s)) and N(t)∼ Poi(λt). Because of the second property
from definition 1.6, N(s) and N(t) − N(s) are independent for 0
\textless{} s \textless{} t, the poisson processes for N(t + s) − N(t)
and N(t) are independent. And the distribution for N(t + s) − N(t)
depends only on s. Subtracting the distributions Poi(λ(t+s))- Poi(λt)
leaves us with N(t + s) − N(t) ∼ Poi(λs).

Lemma 1.7 (Generating Poisson integers based on realizations of
exponential random variables). Generate independent exponential random
realizations e1, e2, . . . from Exp(λ) i.i.d until e1 + · · · + ek+1
\textgreater{} 1 but e1 + · · · + ek ≤ 1 (1.2) then set x := k. Then x
is a realization from Poi(λ).

\begin{Shaded}
\begin{Highlighting}[]
\CommentTok{#I created a modifed version of the exp function for I coded this problem}
\NormalTok{mexp <-}\StringTok{ }\ControlFlowTok{function}\NormalTok{(lambda)\{}
\NormalTok{va<-}\StringTok{ }\KeywordTok{runif}\NormalTok{(}\DecValTok{1}\NormalTok{)}
\NormalTok{var<-}\StringTok{ }\OperatorTok{-}\NormalTok{(}\KeywordTok{log}\NormalTok{(1−va))}\OperatorTok{/}\NormalTok{lambda}
\NormalTok{out <-}\StringTok{ }\NormalTok{var}
\KeywordTok{return}\NormalTok{(out)}
\NormalTok{\}}
\CommentTok{#our PIT generated exponential function}
\NormalTok{mypoi <-}\StringTok{ }\ControlFlowTok{function}\NormalTok{(R,lambda)\{}
\NormalTok{out <-}\StringTok{ }\KeywordTok{rep}\NormalTok{(}\OtherTok{NA}\NormalTok{, R) }\CommentTok{# initialize output vector of length R}
\ControlFlowTok{for}\NormalTok{ (i }\ControlFlowTok{in} \DecValTok{1}\OperatorTok{:}\NormalTok{R)\{}
\NormalTok{  sum<-}\StringTok{ }\DecValTok{0}
\NormalTok{  counter<-}\StringTok{ }\DecValTok{0}
  \ControlFlowTok{while}\NormalTok{(sum}\OperatorTok{<=}\DecValTok{1}\NormalTok{)\{}
\NormalTok{    ek<-}\KeywordTok{mexp}\NormalTok{(lambda)}
\NormalTok{    sum<-}\StringTok{ }\NormalTok{sum}\OperatorTok{+}\NormalTok{ek}
\NormalTok{    counter<-}\StringTok{ }\NormalTok{counter}\OperatorTok{+}\DecValTok{1}
\NormalTok{  \}}
\NormalTok{out[i] <-}\StringTok{ }\NormalTok{counter}\DecValTok{-1}
\NormalTok{\}}
\KeywordTok{return}\NormalTok{(out)}
\NormalTok{\}}


\KeywordTok{set.seed}\NormalTok{(}\DecValTok{1}\NormalTok{)}
\NormalTok{lambd<-}\StringTok{ }\DecValTok{10}


\KeywordTok{system.time}\NormalTok{(mysample <-}\StringTok{ }\KeywordTok{mypoi}\NormalTok{(}\DecValTok{100000}\NormalTok{,lambd))}
\end{Highlighting}
\end{Shaded}

\begin{verbatim}
##    user  system elapsed 
##    4.47    0.04    4.96
\end{verbatim}

\begin{Shaded}
\begin{Highlighting}[]
\NormalTok{(Freq <-}\StringTok{ }\KeywordTok{table}\NormalTok{(mysample))}
\end{Highlighting}
\end{Shaded}

\begin{verbatim}
## mysample
##     0     1     2     3     4     5     6     7     8     9    10    11    12 
##     6    45   231   727  1810  3825  6170  9132 11290 12588 12530 11315  9465 
##    13    14    15    16    17    18    19    20    21    22    23    24    25 
##  7257  5332  3514  2097  1259   695   389   179    79    34    15    10     5 
##    29 
##     1
\end{verbatim}

\begin{Shaded}
\begin{Highlighting}[]
\KeywordTok{hist}\NormalTok{(mysample,}\DataTypeTok{freq=}\OtherTok{FALSE}\NormalTok{,}\DataTypeTok{main=}\StringTok{"Relative Frequencies of Poi(10) Distribution"}\NormalTok{)}
\end{Highlighting}
\end{Shaded}

\includegraphics{compstats1_files/figure-latex/unnamed-chunk-3-1.pdf}

\begin{Shaded}
\begin{Highlighting}[]
\KeywordTok{hist}\NormalTok{(}\KeywordTok{dpois}\NormalTok{(}\DecValTok{1}\OperatorTok{:}\DecValTok{30}\NormalTok{,lambd),}\DataTypeTok{freq=}\OtherTok{FALSE}\NormalTok{,}\DataTypeTok{main=}\StringTok{"Relative Frequencies of truePoi(10) Distribution"}\NormalTok{)}
\end{Highlighting}
\end{Shaded}

\includegraphics{compstats1_files/figure-latex/unnamed-chunk-3-2.pdf}

When you save the notebook, an HTML file containing the code and output
will be saved alongside it (click the \emph{Preview} button or press
\emph{Ctrl+Shift+K} to preview the HTML file).

The preview shows you a rendered HTML copy of the contents of the
editor. Consequently, unlike \emph{Knit}, \emph{Preview} does not run
any R code chunks. Instead, the output of the chunk when it was last run
in the editor is displayed.

\end{document}
